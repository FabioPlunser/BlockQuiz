\documentclass[11pt,t,usepdftitle=false,aspectratio=169]{beamer}

\usepackage[utf8]{inputenc}
\usepackage{graphicx}
\usepackage{listings}
\usepackage[english]{babel}
\usepackage[backend=biber,style=numeric-comp,sorting=none]{biblatex}
\usepackage{filecontents}
\usepackage{xcolor}
\usepackage{multimedia}
\usepackage{animate}


\definecolor{codegreen}{rgb}{0,0.6,0}
\definecolor{codegray}{rgb}{0.5,0.5,0.5}
\definecolor{codepurple}{rgb}{0.58,0,0.82}
\definecolor{backcolour}{rgb}{0.95,0.95,0.92}

\usetheme[nototalframenumber,foot,logo,nosectiontitlepage]{uibk}
\headerimage{3}

\title{Simulation und Schnittstelle mit einer physikalischenn Codeumgebung durch Blockly}
\subtitle{Abetilung für Forschungsgruppe Quality Engineering }
\footertext{Simulation und Schnittstelle mit einer physikalischenn Codeumgebung durch Blockly}

\author[]{Fabio Plunser\\{\footnotesize Betreuer: Michael Vierhauser \& Tobias Antensteiner}}


\addbibresource{bib.bib}

\lstdefinestyle{mystyle}{
    backgroundcolor=\color{backcolour},   
    commentstyle=\color{codegreen},
    keywordstyle=\color{magenta},
    numberstyle=\tiny\color{codegray},
    stringstyle=\color{codepurple},
    basicstyle=\ttfamily\footnotesize,
    breakatwhitespace=false,         
    breaklines=true,                 
    captionpos=b,                    
    keepspaces=true,                 
    numbers=left,                    
    numbersep=5pt,                  
    showspaces=false,                
    showstringspaces=false,
    showtabs=false,                  
    tabsize=2
}

\lstset{style=mystyle}

\begin{document}

\maketitle

\begin{frame}{Inhalt}
    \begin{enumerate}
        \item Motivation
        \item Ziel 
        \item Physische Blöcke
        \item Blockly
        \item Zeitplan
    \end{enumerate}
\end{frame}


\begin{frame}{Motivation}
    Kindern programmieren näher bringen:
    \begin{itemize}
        \item Verknüpfung der physischen Welt mit der digitalen Wert 
        \item Programmieren greifbar machen 
        \item Programmieren durch spielen lernen
    \end{itemize}
\end{frame}

\begin{frame}{Ziel}
    \begin{itemize}
        \item Blöcke werden live auf Webseite dargestellt 
        \item Webseite simuliert den Code
        \item Blöcke haben vorgefertigte funktionen 
        \item Blöcke können von der Webseite Programmiert werden  
    \end{itemize}
    % \begin{center}
    %     \animategraphics[width=0.8\textwidth, loop, autoplay]{15}%frame rate
    %     {turtle/Turtle-}%path to figures
    %     {0}%start index
    %     {58}%end index
    % \end{center}
\end{frame}

    
\begin{frame}{Physische Blöcke}
    Entwickelt von David Rieser in seiner Bachelorarbeit
    \begin{itemize}
        \item Sollen so einfach und günstig wie möglich sein 
        \item Einfacher komplett dynamischer Anschluss mit Magneten 
        \item Sollen programmierbar sein und eine einfache und ausführliche Schnittstelle bereitstellen
    \end{itemize}
    \begin{center}
        \begin{figure}
            \includegraphics[width=0.45\linewidth]{images/bloks.jpg}
            \cite{project_blocks}
        \end{figure}
    \end{center}
\end{frame}


\begin{frame}{Physische Blöcke Challenges}
   \begin{itemize}
     \item Korrekte Programmierung der Blöcke 
     \item Korrekte Interpretation des Blockbaumes 
     \item Korrekte Darstellung der Blöcke in Blocky
     \item Echtzeit Darsetllung der Blöcke
   \end{itemize} 
   Schnittstelle wird kollaborativ mit David Rieser Entwickelt. 
   Über WebUSB
\end{frame}


\begin{frame}{Blockly}
    Blockly is ein visueller Editor von Google. Per drag-and-drop programmieren mit Codegenerierung
    \cite{blockly}.

    \begin{figure}[!htb]
        \centering
        \includegraphics[width=\linewidth]{./images/Blockly.png}
    \end{figure}
\end{frame}

\begin{frame}{Blockly - Challenges}
    \begin{itemize}
        \item Dynamische Block Zusammenstellung basierend auf den physischen Blöcken
        \item Korrektes generieren des Codes, der physischen Blöcke 
        \item Interaktion mit Code beispielen wie Turtle Darstellung 
        \item Gute vordefinierte Blöcke um guten Code zu generieren 
        \item Korrekte ausführung Codes um die Blöcke zu "simulieren"
        \item Einfaches Interface für Nutzer 
    \end{itemize}
\end{frame}


\section{Time Table}
\begin{frame}{Zeitplan}
    \includegraphics[width=\linewidth]{images/Gantt-Chart.pdf}
\end{frame}

\begin{frame}{Danke für die Aufmerksamkeit}
    \includegraphics[width=\linewidth]{_images/uibk_header1.png}
\end{frame}

% \begin{frame}{Verwandte Arbeit}
%     \begin{enumerate}
%         \item Project Bloks\footfullcite{project_blocks}
%         \item Blockly \footfullcite{blockly}
%     \end{enumerate}
% \end{frame}

\begin{frame}[allowframebreaks]
        \frametitle{References}
%        \bibliographystyle{amsalpha}
\printbibliography
%        \bibliography{bib2.bib}
\end{frame}

\end{document}
