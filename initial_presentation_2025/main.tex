\documentclass[11pt,t,usepdftitle=false,aspectratio=169]{beamer}

\usepackage[utf8]{inputenc}
\usepackage{graphicx}
\usepackage{listings}
\usepackage[english]{babel}
\usepackage[backend=biber,style=numeric-comp,sorting=none]{biblatex}
\usepackage{xcolor}
\usepackage{csquotes}
\usepackage{hyperref}

% Colors for code
\definecolor{codegreen}{rgb}{0,0.6,0}
\definecolor{codegray}{rgb}{0.5,0.5,0.5}
\definecolor{codepurple}{rgb}{0.58,0,0.82}
\definecolor{backcolour}{rgb}{0.95,0.95,0.92}

% UIBK theme
\usetheme[nototalframenumber,foot,logo,nosectiontitlepage]{uibk}
\headerimage{3}

% Metadata
\title{BlockQuiz: Web-Based Block Programming Platform with Auto-Grading for students}
\subtitle{Quality Engineering Research Group}
\footertext{BlockQuiz: Web-Based Block Programming Platform}
\author[]{Fabio Plunser\\{\footnotesize Supervisor: Ass. Prof. Dr. Michael Vierhauser}}

% Code listing style (kept)
\lstdefinestyle{mystyle}{
  backgroundcolor=\color{backcolour},
  commentstyle=\color{codegreen},
  keywordstyle=\color{magenta},
  numberstyle=\tiny\color{codegray},
  stringstyle=\color{codepurple},
  basicstyle=\ttfamily\footnotesize,
  breakatwhitespace=false,
  breaklines=true,
  captionpos=b,
  keepspaces=true,
  numbers=left,
  numbersep=5pt,
  showspaces=false,
  showstringspaces=false,
  showtabs=false,
  tabsize=2
}
\lstset{style=mystyle}


\addbibresource{bib.bib}
\begin{document}

\maketitle

% Agenda
\begin{frame}{Outline}
  \begin{enumerate}
    \item Motivation
    \item Requirements
    \item Planned features
    \item Blockly
    \item UI Mockup
    \item Implementation Challenges
    \item Timeline
  \end{enumerate}
\end{frame}

% Motivation
\begin{frame}{Motivation}
  Teaching children programming in an engaging way:
  \begin{itemize}
    \item \textbf{Problem:} Existing tools (Scratch, Tynker, MakeCode) are too open-ended or game/hardware-bound
    \item \textbf{Need:} Focused, auto-graded puzzles with clear learning goals per task
  \end{itemize}
  \vspace{0.3cm}
  \textbf{Solution:} A web-based platform with short, constrained programming exercises inspired by Brilliant.org
\end{frame}

%% Screenshot of brilliant

\begin{frame}{Requirements}
  \begin{itemize}
    \item Customizable block based programming examples
    \item Content management system for courses and exercises
    \item Easy to use interface for students and teachers
    \item Auto-graded, grading support for exercises
    \item Feedback support for students
  \end{itemize}
\end{frame}

% Goal
% \begin{frame}{Goal}
%   Build a web-based, block-programming learning platform for children aged 8--12:
%   \begin{itemize}
%     \item \textbf{Left pane:} Problem statement, examples, hints, feedback
%     \item \textbf{Right pane:} Blockly workspace with constrained toolbox
%     \item \textbf{Interaction:} "RUN" to execute code, "CHECK" for auto-grading
%     \item \textbf{Safety:} Sandboxed execution with timeouts and loop guards
%     \item \textbf{Languages:} English and German (i18n)
%   \end{itemize}
%   \vspace{0.3cm}
%   \textbf{Deliverables:} 10 curated exercises covering loops, conditionals, variables, and simple functions
% \end{frame}

% Platform Concept
\begin{frame}{Planned Features}
  % \textbf{MVP Features:}
  \begin{itemize}
    \item Generated output: Text, code or 2D graphic
    \item Per-exercise constrained block toolbox (reduces cognitive load)
    \item Multi-level hints system
    \item Deterministic auto-grading with public \& hidden test cases
    \item JSON-based import/export for blocks
  \end{itemize}
  \vspace{0.3cm}
  \textbf{Out of Scope:}
  \begin{itemize}
    \item Full classroom management (Kahoot-style)
    \item Complex simulators beyond 2D turtle graphics
  \end{itemize}
\end{frame}

% Blockly
\begin{frame}{Blockly}
  Blockly \cite{blockly} is a visual programming editor from Google
  \begin{itemize}
    \item Drag-and-drop blocks generate executable code
    \item Fully customizable: per-exercise toolbox, block definitions, code generators
    \item Ideal for constrained learning environments
  \end{itemize}
  \vspace{0.3cm}
  \textbf{Why Blockly over MakeCode? \cite{makecode}}
  \begin{itemize}
    \item Library vs. platform: full control over UI and grading logic
    \item Easy integration with SvelteKit and custom CMS
    \item Flexible for quiz-driven workflows
  \end{itemize}
  \begin{figure}[!htb]
    \centering
    \includegraphics[width=0.75\linewidth]{./images/Blockly.png}
  \end{figure}
\end{frame}

% Exercise Types
\begin{frame}{Mockup}
  \centering
  \includegraphics[width=0.85\linewidth]{./images/mockup.pdf}
\end{frame}

\begin{frame}{Implementation Challenges}
  \textbf{Technical:}
  \begin{itemize}
    \item Sandboxed execution with loop detection and timeouts
    \item Deterministic grading with seeded RNG
    \item Multi-language support (i18n) for content and UI
  \end{itemize}
  \vspace{0.3cm}
  \textbf{Design:}
  \begin{itemize}
    \item Defining good exercise criteria (one concept per puzzle, minimal toolbox)
    \item Effective hint system (scaffolding without spoilers)
    \item Age-appropriate feedback and error messages
    \item Balancing expressiveness with cognitive load
  \end{itemize}
\end{frame}

% Timeline
\begin{frame}{Timeline}
  \includegraphics[width=\linewidth]{images/Gantt-Chart.png}
\end{frame}

% Thank you
\begin{frame}{Thank You}
  \begin{center}
    \Large Questions?
  \end{center}
  \vspace{1cm}
  \includegraphics[width=0.8\linewidth]{_images/uibk_header1.png}
\end{frame}

% References
\begin{frame}[allowframebreaks]
  \frametitle{References}
  \printbibliography
\end{frame}

\end{document}